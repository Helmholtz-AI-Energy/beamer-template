\documentclass[aspectratio=1610]{beamer}
% \usepackage[orientation=portrait,size=a2,scale=1.4]{beamerposter}
\usepackage[orientation=portrait,size=a2,scale=1.8]{beamerposter}

% This MUST be imported after beamerposter
\usepackage{helmholtzaiposter} %available: helmholtzai1, helmhotzai2
\usepackage{siunitx}
\sisetup{
    mode=match,
    propagate-math-font=true,
    reset-math-version=false,
    reset-text-family=false,
    reset-text-series=false,
    text-family-to-math=true,
    text-series-to-math=true
}

% Alternative code rendering
\usepackage{minted}

% References
\usepackage[
    backend=biber,
    maxbibnames=2,
    giveninits=true,
    url=true,
    isbn=false,
    sorting=none,
    date=year,
]{biblatex}
\addbibresource{lit.bib}

\title{Helmholtz AI Poster Template}
\author{Dr.\ Con Sultant}
\institute[Helmholtz AI]
  {Helmholtz AI, Germany}
\date{\today}

\begin{document}

\begin{frame}[fragile]{}
    % \vfill
    \vspace{-3ex}
    
    \begin{columns}[t]
        \begin{column}{.5\linewidth}
        
            % First block
            \begin{haiblock}{Example block}

                This template uses a custom environment called
                \texttt{haiblock} which you can use just like a normal LaTeX block
                (it's actually a custom \texttt{tcolorbox})

                \begin{minted}[bgcolor=hgfgray10]{latex}
\begin{haiblock}{Title}
    Contents of block
\end{haiblock}
                \end{minted}


                You can put anything you like inside, for example a figure:
                \begin{figure}
                    \centering
                    \includegraphics[width=0.7\textwidth]{example-image-a}
                \end{figure}
                
            \end{haiblock}
            
            % Column block
            \begin{haiblock}{Block with columns}

            Perhaps you want to compare two things?

            Do so with columns!

            \begin{columns}[t]
                \begin{column}{0.5\linewidth}

                    \textbf{Left column}
                    \begin{itemize}
                        \item You can place
                        \item parallel points
                    \end{itemize}

                    \begin{figure}[b]
                        \centering
                        \includegraphics[width=\columnwidth]{example-image-b}%
                    \end{figure}
                \end{column}%
                \begin{column}{0.5\linewidth}

                    \textbf{Right column}
                    \begin{itemize}
                        \item within their own
                        \item columns
                    \end{itemize}

                    \begin{figure}[b]
                        \centering
                        \includegraphics[width=\columnwidth]{example-image-c}%
                    \end{figure}

                \end{column}
            \end{columns}

            \vspace{1em}

            And then add some additional info
            \begin{figure}
                \centering
                \includegraphics[width=0.45\textwidth]{example-image}%
                \hfill%
                \includegraphics[width=0.45\textwidth]{example-image}%
            \end{figure}
                
            \end{haiblock}
            
        \end{column}
        \begin{column}{.5\linewidth}
        
            % Partners/workflow
            \begin{haiblock}{Block with only figure}

            \begin{figure}
                \centering
                \includegraphics[width=\columnwidth]{example-image}
            \end{figure}
            \end{haiblock}
            
            % Solution
            \begin{haiblock}{Block with code}

                Here's a snippet of code using the minted package

                \begin{minted}[bgcolor=hgfgray10]{python}
import numpy as np

def foo(a, b):
    """
    asd
    """
    return a + b + 1
                \end{minted}

                Wow what a cool function that is!

            \end{haiblock}

            \begin{haiblock}{References}

                \begin{enumerate}
                    \item 
                        Use the usual \texttt{cite} commamd~\cite{baydin_2018}
                    \item 
                        Use \texttt{fullcite} \\
                        \fullcite{baydin_2018}
                    \item 
                        Print the full bibliography
                \end{enumerate}

                \printbibliography
            \end{haiblock}

            
        \end{column}
    \end{columns}
    
    \vfill
    \begin{figure}[b]
        \centering
        \includegraphics[width=0.2\textwidth,height=0.055\textheight,keepaspectratio]{logos/helmholtzai_logo_2_lines.eps}%
        \hfill%
        \includegraphics[width=0.2\textwidth,height=0.055\textheight,keepaspectratio]{logos/helmholtzai_logo_3_lines.eps}%
        \hfill%
        \includegraphics[width=0.2\textwidth,height=0.055\textheight,keepaspectratio]{logos/helmholtzai_logo_2_lines.eps}%
        \hfill%
        \includegraphics[width=0.2\textwidth,height=0.055\textheight,keepaspectratio]{logos/helmholtzai_logo_3_lines.eps}%
    \end{figure}
    
  \end{frame}

 
\end{document}
