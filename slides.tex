%% V0.1
%% 2020/10/09
%% by Markus Götz, Björn Hagemeier, James Kahn

\documentclass[aspectratio=1610]{beamer}
\usepackage{helmholtzai}

\title{Title}
\subtitle{Subtitle}
\author{Firstname Lastname}
\date{YYYY-MM-DD}
\institute{Center}

\begin{document}

\maketitle

\begin{frame}
    \frametitle{Usage}
    
    \begin{enumerate}
        \item Download all files from Github\\~
        \item Edit \texttt{slides.tex} with your favorite editor\\~
        \item Compile the slides by either:\\~
        \begin{enumerate}
            \item Typing \texttt{make} or \texttt{latexmk} in the directory of \texttt{slides.tex} or\\~
            \item Using a LaTeX IDE like TeXstudio\\~
        \end{enumerate}
		\item \emph{Note:} make sure to use LuaLaTeX or XeLaTeX (default in \texttt{make}/\texttt{latexmk}) as compiler 
    \end{enumerate}
\end{frame}

\begin{frame}
    \frametitle{Main Slide Title}
    \framesubtitle{\textbf{Subtitle} with more \emph{details}}
    
    \begin{itemize}
        \item Standard bullet point can be created with the \texttt{itemize} environment
        \item They can have multiple sub-point
        \begin{itemize}
            \item As can be seen here
            \begin{itemize}
                \item Or here
            \end{itemize}
            \item The ordering is unimportant
        \end{itemize}
    \end{itemize}
\end{frame}


\begin{frame}
\frametitle{Equations}

    \begin{equation*}
        f(x) = \sum_i wx_i^2 + \frac{\beta}{2}
    \end{equation*}
\end{frame}


\begin{frame}
    \frametitle{Columns and Figures}

    \begin{columns}
        \begin{column}{0.4\textwidth}
            \begin{enumerate}
                \item Columns allow you to have side-by-side content\\~
                \item Each column itself is its own mini-slide\\~
                \item Figures can be imported by path\\~
                \item Scaling can be done relative to text width, height or initial size
            \end{enumerate}
        \end{column}
        \begin{column}{0.4\textwidth}
            \centering
            \includegraphics[width=\textwidth]{logos/hgf_key_technologies.jpg}
            \source{Helmholtz Association}
        \end{column}
    \end{columns}
\end{frame}

\begin{frame}[fragile]
    \frametitle{Code}
    
    \emph{Note the [fragile] specifier next to frame and the code indentation.}
    
\begin{lstlisting}[language=Python]
import numpy as np

def foo(a, b):
    """
    asd
    """
    return a + b + 1
\end{lstlisting}
\end{frame}

% command for making the color boxes
\newcommand\crule[3][black]{\textcolor{#1}{\rule{#2}{#3}}}

\begin{frame}
    \frametitle{Colors}
    \framesubtitle{Basic Definitions}
    
    \emph{The beamer template contains definitions for all Helmholtz colors.}\\
    
    \begin{table}
        \centering
        \small
        \begin{tabular}{cl}
            \textbf{Color} & \textbf{Name}\\\toprule
            \crule[hgfblue]{10pt}{10pt} & hgfblue \\
            \crule[hgfdarkblue]{10pt}{10pt} & hgfdarkblue \\
            \crule[hgfgreen]{10pt}{10pt} & hgfgreen \\
            \crule[hgfgray]{10pt}{10pt} & hgfgray \\
            \crule[hgfaerospace]{10pt}{10pt} & hgfaerospace (short: hgfast) \\
            \crule[hgfearthandenvironment]{10pt}{10pt} & hgfearthandenvironment (short: hgfee) \\
            \crule[hgfenergy]{10pt}{10pt} & hgfenergy \\
            \crule[hgfhealth]{10pt}{10pt} & hgfhealth \\
            \crule[hgfkeytechnologies]{10pt}{10pt} & hgfkeytechnologies (short: hgfkt, hgfinformation) \\
            \crule[hgfmatter]{10pt}{10pt} & hgfmatter \\\bottomrule
        \end{tabular}
    \end{table}
\end{frame}

\begin{frame}
    \frametitle{Colors}
    \framesubtitle{Shades}
    
    \emph{For each color there exist 10 lighter shades, exemplary for hgfblue}\\
    
    \begin{table}
        \centering
        \small
        \begin{tabular}{cl}
            \textbf{Color} & \textbf{Name}\\\toprule
            \crule[hgfblue10]{10pt}{10pt} & hgfblue10 \\
            \crule[hgfblue20]{10pt}{10pt} & hgfblue20 \\
            \crule[hgfblue30]{10pt}{10pt} & hgfblue30 \\
            \crule[hgfblue40]{10pt}{10pt} & hgfblue40 \\
            \crule[hgfblue50]{10pt}{10pt} & hgfblue50 \\
            \crule[hgfblue60]{10pt}{10pt} & hgfblue60 \\
            \crule[hgfblue70]{10pt}{10pt} & hgfblue70 \\
            \crule[hgfblue80]{10pt}{10pt} & hgfblue80 \\
            \crule[hgfblue90]{10pt}{10pt} & hgfblue90 \\
            \crule[hgfblue]{10pt}{10pt} & hgfblue \\\bottomrule
        \end{tabular}
    \end{table}
\end{frame}

\begin{frame}
    \frametitle{Blocks}
    
    \begin{block}{block}
        This is how a regular block looks like
    \end{block}
    \vspace{2em}
    \begin{exampleblock}{exampleblock}
        An example block is stilled differently.
    \end{exampleblock}
    \vspace{2em}
    \begin{alertblock}{alertblock}
        Alert blocks can draw attention to critical information
    \end{alertblock}
\end{frame}

\begin{frame}
    \frametitle{Special Formatting}
    
    \begin{itemize}
        \item There are raw links with the full URL \url{https://www.google.com}
        \item You can add also links with names \href{https://www.google.com}{Google}\\~
        \item You might also want to write in \hermann{Hermann Bold} - Helmholtz's title font
    \end{itemize}
\end{frame}

\section{Sections look like this}

\end{document}
